\documentclass[10pt, a4paper]{article}

\usepackage[utf8]{inputenc}  
\usepackage[T1]{fontenc}  
\usepackage[francais]{babel}

\usepackage{graphicx}
\usepackage{xspace}
\usepackage{subfig}
\usepackage{amsmath}

\usepackage{geometry}
\geometry{hmargin=3.5cm,vmargin=2.5cm}

%%%%%%%%%%%%%%%%%%%%%%%%%% Notations %%%%%%%%%%%%%%%%%%%%%%%%%%
\newcommand{\A}{A} % matrice complète
\newcommand{\B}{B} % matrice incomplète
\newcommand{\Ap}{\widehat{A}} % matrice approximée

\newcommand{\Ac}[2]{a_{#1,#2}} % matrice complète composante
\newcommand{\Bc}[2]{b_{#1,#2}} % matrice incomplète composante
\newcommand{\Apc}[2]{\widehat{a}_{#1,#2}} % matrice approximée composante

\newcommand{\rmse}[1]{\mathrm{RMSE}\pp{#1}} % root mean square error
\newcommand{\norme}[2]{\left\lVert #1 \right\rVert_{#2}} % norme
\newcommand{\tnorme}[2]{\left|\hspace{-1pt}\left\lVert #1 \right\rVert\hspace{-1pt}\right|_{#2}} % norme

\newcommand{\tr}[1]{#1^{\mathrm{T}}} % Trace
\newcommand{\ei}[1]{\mathrm{e}_i} % Vecteur de base

%raccoursis
\newcommand{\pp}[1]{\left(#1\right)} % parenthèses
\newcommand{\nl}{

~

} % saute de line
%%%%%%%%%%%%%%%%%%%%%%%%%% Notations %%%%%%%%%%%%%%%%%%%%%%%%%%

\begin{document}

\title{Recommandation - Rapport partiel}
\author{Ken Chanseau--Saint-Germain \& Vincent Vidal}
\date{\today}
\maketitle

\section*{Notation}
On notera pour $p>0$ un réel, $X$ un vecteur et $M$ une matrice quelconque : \[
	\norme{X}{p} = \pp{\sum_i \left|x_i\right|^p}^{\frac{1}{p}} \hspace{2cm} 
	\norme{M}{p} = \pp{\sum_{i, j} \left|m_{i,j}\right|^p}^{\frac{1}{p}}
\]\[
	\tnorme{M}{p} = \sup_{\norme{x}{p}=1} \norme{Mx}{p}
\]
Et on posera $\norme{X}{0}$ le nombre de composantes non nulles de $X$. 

\section{Introduction}

On se donne ici $n$ personnes donnant des notes à $m$ objets.\newline
On notera $\Ac{i}{j}$ la note de l'individu $i$ sur l'objet $j$ ainsi que $A = \pp{\Ac{i}{j}}_{\substack{1\leq i\leq n \\ 1 \leq j \leq m}}$ la matrice des notes.

On suppose ici que l'on a accès qu'à une matrice incomplète $\B$ obtenu en annulant certaines composantes de $A$. Le but est alors de trouver une bonne approximation $\Ap$ de $A$ à partir de $\B$.

\nl

On prendra comme mesure d'approximation, l'erreur moyenne suivante :
\[
	\rmse{\Ap} = \norme{A-\Ap}{2} = \sqrt{\sum_{i,j} \pp{\Ac{i}{j} - \Apc{i}{j}}^2}
\]

\section{Approximation basique}

On considère ici l'approximation suivante : \[
\Apc{i}{j} = m_{\Ap} + p_i + o_j
\]
Avec $m_{\Ap}$ la moyenne des valeurs de $\Ap$, $p_i$ la moyenne des notes recentrées qu'a donné la personne $i$ et $o_j$ la moyenne des notes recentrées qu'a obtenu l'objet $j$. C'est à dire : \[
	m_{\Ap} = \frac{\sum_{i,j}\Bc{i}{j}}{\norme{\B}{0}}  \hspace{1cm}
	p_i = \frac{\sum_{j}\Bc{i}{j} - m_{\Ap}}{\norme{\tr{\B}\ei{i}}{0}} \hspace{1cm}
	o_j = \frac{\sum_{i}\Bc{i}{j} - m_{\Ap}}{\norme{\B\ei{j}}{0}}
\]




\end{document}
